\documentclass[
  xcolor={svgnames},
  hyperref={colorlinks,citecolor=DeepPink4,linkcolor=DarkRed,urlcolor=DarkBlue}
  ]{beamer}

% define using customized theme.
\usetheme{pas}

% define using packages
\usepackage[utf8]{inputenc}
\usepackage[T1]{fontenc}

% the general information.
\title[Largest Prime Number?] % (optional, only for long titles)
{There Is No Largest Prime Number}
\subtitle{Presents with Stuttgart Pas Beamer theme}

\author[tmip, hieutt] % (optional, for multiple authors)
{T.~Mip\inst{1} \and TT.~Hieu\inst{2}}
\institute[Universities Here and There] % (optional)
{
  \inst{1}%
  Institute of Computer Science\\
  University Here
  \and
  \inst{2}%
  Institute of Theoretical Philosophy\\
  University There
}
\date[TX 2014] % (optional)
{Base on a guide on \cite{BaseOn},2014}
\subject{Computer Science}



% begin presentation content
\begin{document}

\begin{frame}
\titlepage
\end{frame}


\begin{frame}
\frametitle{There Is No Largest Prime Number}
\framesubtitle{The proof uses \textit{reductio ad absurdum}.}
\begin{theorem}
There is no largest prime number. \end{theorem}
\begin{enumerate}
\item<1-| alert@1> Suppose $p$ were the largest prime number.
\item<2-> Let $q$ be the product of the first $p$ numbers.
\item<3-> Then $q+1$ is not divisible by any of them.
\item<1-> But $q + 1$ is greater than $1$, thus divisible by some prime
number not in the first $p$ numbers.
\end{enumerate}
\end{frame}

\begin{frame}{A longer title}
\begin{itemize}
\item one
\item two
\end{itemize}
\end{frame}


\begin{frame}[allowframebreaks]
   \frametitle{References}
   \bibliographystyle{plain}
   \bibliography{lib}
\end{frame}

\end{document}
